\documentclass[11pt]{article}

% Packages for better formatting and math support
\usepackage[T1]{fontenc}
\usepackage[utf8]{inputenc}
\usepackage{amsmath, amssymb, amsthm}
\usepackage{enumitem}
\usepackage{hyperref}
\usepackage{lipsum} % for dummy text
\usepackage{fancyhdr}
\usepackage{titlesec}

% Page layout
\usepackage{geometry}
\geometry{a4paper, margin=1in}
\addtolength{\topmargin}{-2pt}

% Header and footer
\setlength{\headheight}{14pt}
\pagestyle{fancy}
\fancyhf{}
\fancyhead[L]{\leftmark}
\fancyhead[R]{\rightmark}
\fancyfoot[C]{\thepage}

% Section formatting
\titleformat{\section}
  {\normalfont\Large\bfseries}{\thesection}{1em}{}
\titleformat{\subsection}
  {\normalfont\large\bfseries}{\thesubsection}{1em}{}
\titleformat{\subsubsection}
  {\normalfont\normalsize\bfseries}{\thesubsubsection}{1em}{}

% Theorem, Definition, and Example environments
\newtheorem{theorem}{Theorem}[section]
\newtheorem{definition}[theorem]{Definition}
\newtheorem{example}[theorem]{Example}
\newtheorem{remark}[theorem]{Remark}

% Custom commands for easier math notation
\newcommand{\R}{\mathbb{R}}
\newcommand{\N}{\mathbb{N}}
\newcommand{\Z}{\mathbb{Z}}
\newcommand{\Q}{\mathbb{Q}}
\newcommand{\C}{\mathbb{C}}

\title{Regression}
\author{Syed Arham Naqvi}
\date{\today}

\begin{document}

\maketitle
\tableofcontents
\newpage

\section{Introduction}
These notes will contain a breif overview of the following topics:
\begin{itemize}
    \item Linear Regression
    \item Locally Weighted Regression
    \item Logistic Regression
\end{itemize}
Regression 

\newpage

\section{Key Concepts}

\subsection{Definition and Theorems}
\begin{definition}[Limit of a Sequence]
    Let $\{a_n\}$ be a sequence of real numbers. We say that $a_n$ converges to $L \in \R$ if for every $\epsilon > 0$, there exists an $N \in \N$ such that for all $n > N$, $|a_n - L| < \epsilon$.
\end{definition}

\begin{theorem}[Fundamental Theorem of Algebra]
    Every non-constant polynomial with complex coefficients has at least one complex root.
\end{theorem}

\subsection{Examples and Applications}
\begin{example}
    Consider the sequence $\{a_n\} = \frac{1}{n}$. This sequence converges to $0$ as $n$ approaches infinity.
\end{example}

\subsection{Notes and Remarks}
\begin{remark}
    The Fundamental Theorem of Algebra implies that a polynomial of degree $n$ has exactly $n$ roots in the complex plane, counting multiplicities.
\end{remark}

\section{Advanced Topics}

\subsection{Differential Equations}
\begin{itemize}
    \item Introduction to differential equations.
    \item First-order differential equations.
    \begin{itemize}
        \item Separable equations.
        \item Linear equations.
    \end{itemize}
    \item Higher-order differential equations.
\end{itemize}

\subsection{Linear Algebra}
\begin{itemize}
    \item Vector spaces.
    \item Linear transformations.
    \item Eigenvalues and eigenvectors.
    \begin{itemize}
        \item Definition and properties.
        \item Applications in solving systems of linear equations.
    \end{itemize}
\end{itemize}

\section{Conclusion}
\begin{itemize}
    \item Summary of the main points.
    \item Potential areas for further study.
\end{itemize}

\bibliographystyle{plain}
\bibliography{references}

\end{document}
